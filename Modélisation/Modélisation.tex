\documentclass[11pt,a4paper]{article}


\usepackage[utf8]{inputenc}
\usepackage[T1]{fontenc}
\usepackage[french]{babel}
\usepackage{amsmath,amsthm,amssymb}
\usepackage{algorithm}
\usepackage{algpseudocode}
\usepackage{graphicx}
\usepackage{listings}
\usepackage{hyperref}
\usepackage{xcolor} 
\usepackage[most]{tcolorbox}
\usepackage{float}
\usepackage{tikz}
\usetikzlibrary{trees}
\usepackage{fancyhdr}
\pagestyle{fancy}
\usepackage{relsize}

\fancyhf{}

\renewcommand{\headrulewidth}{0.4pt}
\fancyhead[R]{\small M1 INFO}

\renewcommand{\footrulewidth}{0.4pt}
\fancyfoot[L]{ARP}
\fancyfoot[R]{\thepage}

\begin{document}

\title{Algorithmique de résolution de problèmes: Bipartite-Subgraph Problem}
\author{J. KOZIK}
\date{03-12-2025}

\maketitle

\newpage

\begin{flushleft}

\section{Modélisation}


\subsection{Etat}

un etat décrit:
\begin{itemize}
\item[$\blacktriangleright$] Deux ensembles $E_c$ (ensemble d'arêtes restantes) et $E_r$ (ensemble d'arêtes retirées).
\end{itemize}

\subsubsection{Etat initial}

A l'état initial, l'ensembles $E_c$ contient toutes les arêtes initiales, et $E_r$ est vide.
\subsubsection{Successeur}

Pour un état $e$:\\
$succ(Etat,e) =e^{'}$ telle que on ajoute $e$ à $E_r$ et on l'enlève de $E_c$.

\smallskip



\subsubsection{Etat valide}

A tout moment, un état est valide si il représente un sous-graphe du problème initial.

\subsubsection{Etat final}

un état $e$ est final si le graphe $G_f = (V,E_c)$ obtenues est biparti. 
\end{flushleft}
\end{document}
 