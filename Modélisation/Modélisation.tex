\documentclass[11pt,a4paper]{article}


\usepackage[utf8]{inputenc}
\usepackage[T1]{fontenc}
\usepackage[french]{babel}
\usepackage{amsmath,amsthm,amssymb}
\usepackage{algorithm}
\usepackage{algpseudocode}
\usepackage{graphicx}
\usepackage{listings}
\usepackage{hyperref}
\usepackage{xcolor} 
\usepackage[most]{tcolorbox}
\usepackage{float}
\usepackage{tikz}
\usetikzlibrary{trees}
\usepackage{fancyhdr}
\pagestyle{fancy}
\usepackage{relsize}

\fancyhf{}

\renewcommand{\headrulewidth}{0.4pt}
\fancyhead[R]{\small M1 INFO}

\renewcommand{\footrulewidth}{0.4pt}
\fancyfoot[L]{ARP}
\fancyfoot[R]{\thepage}

\begin{document}

\title{Algorithmique de résolution de problèmes: Bipartite-Subgraph Problem}
\author{J. KOZIK}
\date{03-12-2025}

\maketitle

\newpage

\section{Modélisation}

\subsection{État}

Un état est défini par deux ensembles d’arêtes :
\begin{itemize}
\item[$\blacktriangleright$] $E_c \subseteq E$, l’ensemble des arêtes conservées ;
\item[$\blacktriangleright$] $E_r \subseteq E$, l’ensemble des arêtes retirées.
\end{itemize}

Ces ensembles vérifient les invariants suivants :

$E_c \cup E_r = E \quad \text{et} \quad E_c \cap E_r = \varnothing.$


\subsubsection{État initial}

À l’état initial, toutes les arêtes sont conservées :

$E_c = E \quad \text{et} \quad E_r = \varnothing.$


\subsubsection{Successeur}

Soit un état $s = (E_c, E_r)$ et une arête $e \in E_c$.  
Le successeur $s'$ obtenu par le retrait de $e$ est défini par :

$E_c' = E_c \setminus \{e\}, \quad E_r' = E_r \cup \{e\}.$

\subsubsection{État valide}

Tout état respectant les invariants précédents est considéré comme valide, car il représente un sous-graphe du graphe initial.

\subsubsection{État final}

Un état $s$ est final si le graphe induit $G_f = (V, E_c)$ est biparti, c’est-à-dire s’il ne contient aucun cycle impair.


\end{document}
 