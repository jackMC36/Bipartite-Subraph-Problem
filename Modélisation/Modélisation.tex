\documentclass[11pt,a4paper]{article}


\usepackage[utf8]{inputenc}
\usepackage[T1]{fontenc}
\usepackage[french]{babel}
\usepackage{amsmath,amsthm,amssymb}
\usepackage{algorithm}
\usepackage{algpseudocode}
\usepackage{graphicx}
\usepackage{listings}
\usepackage{hyperref}
\usepackage{xcolor} 
\usepackage[most]{tcolorbox}
\usepackage{float}
\usepackage{tikz}
\usetikzlibrary{trees}
\usepackage{fancyhdr}
\pagestyle{fancy}
\usepackage{relsize}

\fancyhf{}

\renewcommand{\headrulewidth}{0.4pt}
\fancyhead[R]{\small M1 INFO}

\renewcommand{\footrulewidth}{0.4pt}
\fancyfoot[L]{ARP}
\fancyfoot[R]{\thepage}

\begin{document}

\title{Algorithmique de résolution de problèmes: Bipartite-Subgraph Problem}
\author{J. KOZIK}
\date{03-12-2025}

\maketitle

\newpage

\begin{flushleft}

\section{Modélisation}


\subsection{Etat}

un etat décrit:
\begin{itemize}
\item[$\blacktriangleright$] Deux ensembles $X$ et $Y$ de sommets. 
\end{itemize}

\subsubsection{Etat initial}

A l'état initial, les ensembles $X$ et $Y$ sont vides.

\subsubsection{Successeur}

Pour un état $e$:\\
$succ(e,v,X) =e^{'}$ telle que on ajoute $v$ à $X$.

\smallskip

$succ(e,v,Y) =e^{'}$ telle que on ajoute $v$ à $Y$.




\subsubsection{Etat valide}

A tout moment, un état est valide si:
\begin{itemize}
\item[$\blacktriangleright$] $\forall v \in V$, $v$ appartient exclusivement à $X$ ou $Y$ ou à aucun des deux.
\item[$\blacktriangleright$] le sous-graphe induit par $(X,E)$ forme un stable.
\item[$\blacktriangleright$] le sous graphe induit par $(Y,E)$ forme un stable.
\end{itemize}

\subsubsection{Etat final}

un état $e$ est final si:\\
\begin{itemize}
\item[$\blacktriangleright$]$X \cup Y = V$
\item[$\blacktriangleright$] $e$ est un état valide.
\end{itemize}

\end{flushleft}
\end{document}
 